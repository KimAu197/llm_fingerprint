\documentclass{article}

\usepackage[final]{neurips_2019}

\usepackage[utf8]{inputenc}
\usepackage[T1]{fontenc}
\usepackage{hyperref}
\usepackage{url}
\usepackage{booktabs}
\usepackage{amsfonts}
\usepackage{nicefrac}
\usepackage{microtype}
\usepackage{graphicx}
\usepackage{xcolor}
\usepackage{lipsum}

\newcommand{\note}[1]{\textcolor{blue}{{#1}}}

\title{
  Title of your project \\
  \vspace{1em}
  \small{\normalfont COMS 4705 Project Proposal}  \\
  \small{\normalfont \textbf{Keywords:} \textit{keyword1, keyword2, ...}}

}

\author{
  Name \\
  Department of Computer Science \\
  Columbia University \\
  \texttt{UNI@columbia.edu} \\
  \And
  Name \\
  Department of Computer Science \\
  Columbia University \\
  \texttt{UNI@columbia.edu} \\
% Examples of more authors
%   \And
%   Name \\
%   Department of Computer Science \\
%   Columbia University \\
%   \texttt{UNI@columbia.edu}
}

\begin{document}

\maketitle

% \begin{abstract}
%   Required for final report
% \end{abstract}

\begin{center}
    \note{This template is built on NeurIPS 2019 template\footnote{\url{https://www.overleaf.com/latex/templates/neurips-2019/tprktwxmqmgk}} and adapted from Stanford CS224N Natural Language Processing with Deep Learning 
    }
\end{center}


\section{Key Information to include}

\begin{itemize}
    \item External collaborators (if you have any):
    \item Mentor:
    \item Sharing project:
\end{itemize}

\section{AI-aided literature review and critique}
\textit{.5-1 page excluding literature review, 25 points}

    \paragraph{Chosen LLM \& Responses.} Also report the LLM you prompted, a copy of the LLM's response, and a copy of a section of the ScholarQA report into your proposal.

    \paragraph{Question.} Report the question you chose and a brief motivation for why you chose it. For example, why is this an interesting question to ask? Why would investigating this question be important?. 

    \paragraph{Critique of Literature Review.} Summarize the strengths and weaknesses of the responses that you notice. Do they address the question you intended them to answer? Does either response seem to be missing any key related topics or details? Which do you find more helpful and why? 
    
    \paragraph{Chosen Article.} Report and cite the article you selected to read. Include a \textbf{brief} (2-3 sentence) summary of what primary question the article aimed to study and the central findings. Was this particular article relevant to your question (why or why not)? What does it contribute in the context of the other referenced articles? Was this article appropriately discussed in the response you drew from (why or why not)? 
    
    \paragraph{Reflection.} Discuss what you gained from reviewing these responses and the article you chose. Was there any particular approach or task mentioned that you found interesting? Was the answer or partial answer provided what you expected? What gaps in research or open questions remain that you notice? It is okay if you didn't find the responses helpful or interesting, but you should discuss why not.

\section{Project description}
\textit{1-2 pages, 5 points}

    \paragraph{Goal.} 
        If possible, try to phrase this in terms of a scientific question or questions you are trying to answer -- e.g., your goal may be to investigate whether a particular model or technique performs well at a certain task, or whether you can improve a particular model by adding some new variant, or (for theoretical/analytical projects), you might have some particular hypothesis that you seek to confirm or disprove.
        Otherwise, your goal may be simply to successfully implement a complex neural model, and show that it performs well on a given task.
        Briefly motivate why you chose this goal -- why do you think it is important, interesting, challenging and/or likely to succeed?
        If you have any secondary or stretch goals (i.e. things you will do if you have time), please also describe them.
        In this section, you should also make it clear how your project relates to your chosen paper.

    \paragraph{Task.} 
        This could be 
        similar to the question you posed in your literature review or from your chosen paper, but it doesn't have to be. Describe the task clearly (i.e., give an example of an input and an output, if applicable).
    
    \paragraph{Data.}
        Specify the dataset(s) you will use (including its size), and describe any preprocessing you plan to do. If you plan to collect your own data, describe how you will do that and how long you expect it to take.
    
    \paragraph{Methods.}
     Describe the models and/or techniques you plan to use.
            If it's already described in the paper summary, no need to repeat.
            If you plan to explore a variant to a published method, focus on describing how your method will be different.
            Particularly if you are planning to replicate an existing paper, make it clear which parts you plan to implement yourself, and which parts you will download from elsewhere. 
            If there is any part of your planned method that is original, make it clear. If you are proposing a new approach or change to an existing approach, be sure to briefly motivate why you think this will work or what benefits it offers (e.g., better performance for a given metric, more efficient, etc.).
    
    \paragraph{Baselines.}
        Describe what methods you will use as baselines. 
        Baselines contextualize your results to show how your approach improves on prior or simpler approaches. For example, if you make changes to an existing model finetuned for a given task, you should evaluate the original model to show how your change impacts performance.
        Make it clear if these will be implemented by you, downloaded from elsewhere, or if you will just compare with previously published scores. Be sure to also briefly motivate why you chose specific baselines. For example, are they the best performing baselines comparable to your approach? What insights does comparing your proposed approach to these baselines offer?
    
    \paragraph{Evaluation.}
        Specify at least one well-defined, numerical, automatic evaluation metric you will use for quantitative evaluation. Note that often, it is appropriate to employ multiple evaluation metrics that can account for each others' weaknesses, but you need only report one in your proposal. Justify why your chosen metric or metrics are appropriate for what you intend to measure.
        What existing scores will you be comparing against for this metric? For example, if you're reimplementing or extending a method, state what score(s) the original method achieved; if you're applying an existing method to a new task, mention the state-of-the-art performance on the new task, and say something about how you expect your method to perform compared to other approaches.
        If you have any particular ideas about the qualitative evaluation you will do, you can describe that too.
    
    \paragraph{Justification}
        Briefly (a couple of sentences) justify how your proposal satisfies the expectations for the final project (see "What to Expect" in the proposal guidelines). Additionally, describe what each team member will tentatively be responsible for (this can change as the project develops) and why you believe it is feasible to complete by the end of the semester.

    \bibliographystyle{unsrt}
    \bibliography{references}

\end{document}
